Refinements to Month 1 :
Python + PyTest + Requests: Keep it practical. Instead of just a “framework,” show a realistic use case (e.g., testing a mock e-commerce API). Hiring managers want proof you can solve real problems.
Playwright: Showcase cross-browser tests, parallel runs, and integration with PyTest. Highlight this in your GitHub README.
CI/CD: GitLab is fine, but add GitHub Actions too. Many recruiters will look at GitHub first. Even a basic workflow file adds credibility.

✅ Deliverable by end of Month 1: Public GitHub repo with API + UI tests running automatically on each commit, results visible in pipeline artifacts.

___________________________________________________________________________________________________________________________________

Refinements to Month 2 :
Terraform + AWS: Focus on IAM (users, policies, roles) + VPC networking basics. These are the bread-and-butter interview questions.
Monitoring (Prometheus + Grafana): Keep it minimal: scrape a Python app’s metrics and visualize them. Show a working dashboard in your README (screenshots/GIFs).
K6 (Performance Testing): Wrap it into your CI pipeline so load tests run automatically. That’s a nice “extra.”

✅ Deliverable by end of Month 2: GitHub repo: Terraform deploys AWS infra, app is monitored with Prometheus + Grafana, basic load tests run in CI/CD.

____________________________________________________________________________________________________________________________________

Refinements to Month 3
Instead of just “projects,” create a portfolio hub repo (like a landing page). Example: awesome-devops-portfolio with links to:
Test automation project (API + UI)
Terraform AWS project (infra deploy)
K8s monitoring project (Prometheus + Grafana)
Each repo should have a clean README.md with:
What it does
Tools used
Setup steps (so it’s reproducible)
Screenshots/GIFs

✅ Deliverable by end of Month 3: A portfolio that looks like a candidate with experience, not just study projects.